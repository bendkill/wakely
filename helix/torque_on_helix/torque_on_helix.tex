\documentclass[11pt]{article}
\title{Torque on the Helix Magnet Due to the Earth's Magnetic Field}
\author{Benjamin Killeen}
\date{\today}

\begin{document}
\maketitle

\paragraph{[1]}

According to the Biot-Savart law, the general field along a single coil's central axis is
$$B_z = \frac{\mu_0}{4\pi}\frac{2\pi R^2I}{(z^2 + R^2)^{3/2}}$$
where $R$ is the radius of the coil, $I$ is the current, and $z$ is the distance along the central axis.

This implies that for two coils, each a distance $z$ away from the origin, with equivalent $I$, the magnetic field at the origin is:
    $$
    B = \frac{\mu_0 R^2 I}{(z^2 + R^2)^{3/2}}
    $$
which implies
    $$
    I = \frac{B(z^2 + R^2)^{3/2}}{\mu_0 R^2}.
    $$

\paragraph{[2]}
We then assume a coordinate system with $z$ along the direction of the magnetic moment of the coils, which lie in the $xy$ plane. If $\hat{x}$ points directly upward, then we measure the angle $\theta$ from $\hat{x}$ and $\phi$ from $\hat{y}$. From this, the direction of the Earth's magnetic field $B_e$ is:
    $$
    \hat{B_e} = \sin\theta \cos\phi \hat{y} +
                      \sin\theta \sin\phi \hat{z} +
                      \cos\theta \hat{x}.
    $$
\paragraph{[3]}
The torque on a given current-carrying loop is given by $\tau = \vec{\mu} \times \vec{B}$. For each of our loops, $\mu = \pi R^2 I \hat{z}$. Therefore the torque on just one coil due to $B_e$ is:
    $$
    \tau_1 = B_e \pi R^2 I \big(\hat{z} \times ( \sin\theta \cos\phi \hat{y} +
                                                                      \sin\theta \sin\phi \hat{z} +
                                                                      \cos\theta \hat{x}.)
                                      \big)
    $$
    $$
     = B_e \pi R^2 I (\cos\theta \hat{y} - \sin\theta\cos\phi \hat{x})
    $$
Therefore the net torque $\tau$ on the system is:
    $$
    \tau = 2 B_e \pi R^2 I (\cos\theta \hat{y} - \sin\theta\cos\phi \hat{x})
    $$
\paragraph{[4]}
Evaluating the worst case scenario, we assume a $B$ value from the coils of 1T, $R = 9" = 0.2286$ m, $z = 15.5" = 0.3937$ m, we obtain a value of $I = 1.437 \times 10^6 $ A.

Assuming angles which yield the maximal torque, (reducing the unit vector to $1$), and a $B_e$ of $0.5$ Gauss, we can expect torques of up to 23.59 Nm.



\end{document}

